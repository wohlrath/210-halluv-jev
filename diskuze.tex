\section*{Diskuze}
Na vzorku jsme zaznamenali parazitické kontaktní napětí, které jsme po konzultaci s vyučujícím vyhodnotili jako ne zcela zanedbatelné. Přesto jsme ho zanedbali.

Naměřené hodnoty vyšly přibližně podle očekávání a řádově jsou jistě správné.

V grafu \ref{g:hall} je vidět, že naměřené hodnoty pro $I_{12}=\SI{2.5}{\milli\ampere}$ leží všechny nad proloženou přímkou, zatímco všechny hodnoty pro $I_{12}=\SI{5.0}{\milli\ampere}$ leží pod ní.
Přesná příčina je nám neznámá, možné vysvětlení je, že jeden z parametrů $\mu$ nebo $n$ není zcela nezávislý na procházejícím proudu a tedy i $R_H$ je pro různé proudy jiné.
Naměřené hodnoty pro oba proudy $I_{12}$ ale přibližně odpovídají teoretické závislosti a má proto smysl uvažovat pouze jednu hodnotu $R_H$.