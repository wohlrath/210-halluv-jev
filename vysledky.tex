\section*{Výsledky měření}
Měření proběhlo při normálním tlaku a pokojové teplotě (přibližně \SI{22}{\degreeCelsius}).
Všechny uvedené nejistoty jsou standardní a v zápisu \num{10(1)} znamená číslo v závorce nejistotu v řádu poslední uvedené číslice.


Proud $I_{12}$ jsme měřili digitálním multimetrem MASTECH MY-68 a napětí $U_{34}$ a $U_{56}$ digitálním multimetrem METEX MXD 4660A.
Proud $I_M$ procházející elektromagnetem jsme měřili analogovým ampérmetrem s třídou přesnosti \num{0.5} a rozsahem \SI{6}{\ampere}.

Vzorek germania měl rozměry $l=\SI{6.000(5)}{\mm}$, $d=\SI{3.350(5)}{\mm}$ a $t=\SI{0.720(5)}{\mm}$.


Naměřená voltampérová charakteristika je uvedena v tabulce \ref{t:vodivost} a zanesena do grafu \ref{g:vodivost}.

Pomocí programu GNUPLOT 4.6 jsme lineární regresí určili konstantu úměrnosti mezi napětím $U_{34}$ a proudem $I_{12}$ jako \SI{2.124(4)}{\milli\ampere\per\volt}.
Uvedená chyba je pouze statistická (chyba fitu).
Porovnáním s \eqref{e:vodivost} a metodou přenosu chyb jsme vypočítali měrnou vodivost vzorku $\sigma = \SI{5.28(5)}{\siemens\per\meter}$, přičemž jsme odhadli vliv chyby původních měřených veličin na konečný výsledek a zohlednili ho. 



Magnetické pole buzené elektromagnetem mělo indukci
\begin{equation}
B(T)=\num{0.098} \cdot I_M(A)
\end{equation}

Hallovu konstantu jsme změřili pro dva různé proudy vzorkem --- \SI{2.50(6)}{\milli\ampere} a \SI{5.00(9)}{\milli\ampere}. Označme je $R_{H1}$ a $R_{H2}$ resp. Teoreticky by se měly obě hodnoty shodovat.

Naměřené hodnoty jsou uvedeny v tabulce \ref{t:hall} a zaneseny do grafu \ref{g:hall}.

Opět lineární regresí v programu GNUPLOT 4.6 jsme vypočítali konstanty úměrnosti mezi magnetickou indukcí $B$ a Hallovým napětím $U_H$ jako \SI{223(2)}{\milli\volt\per\tesla} pro proud vzorkem \SI{2.5}{\milli\ampere} a \SI{420(2)}{\milli\volt\per\tesla} pro \SI{5}{\milli\ampere}.
Porovnáním s \eqref{e:hall} a metodou přenosu chyb jsme vypočítali Hallovy konstanty $R_{H1}=\SI{0.064(3)}{\metre\cubed\per\ampere\per\second}$ a $R_{H2}=\SI{0.060(3)}{\metre\cubed\per\ampere\per\second}$, přičemž jsme opět odhadli vliv chyby měřených veličin.

Obě hodnoty se v rámci chyby shodují, má tedy smysl uvažovat skutečnou $R_H=\SI{0.062(3)}{\metre\cubed\per\ampere\per\second}$ jako jejich průměr.


Pro tuto hodnotu $R_H$ jsme podle \eqref{eq:pohyblivost} vypočítali Hallovskou pohyblivost $\mu = \SI{0.33(2)}{\ampere\second\squared\per\kg}$ a podle \eqref{eq:koncentrace} koncentraci nositelů náboje $n = \num{1.18(6)} \cdot \SI{e20}{\per\metre\cubed}$.
Chybu jsme určili metodou přenosu chyb.

\begin{tabulka}[htbp]
\centering
\begin{tabular}{c|c}
$U_{34}$ (\si{\volt}) & $I_{12}$ (\si{\milli\ampere}) \\ \hline
\num{0.232(2)} & \num{0.50(4)} \\
\num{0.474(2)} & \num{1.00(5)} \\
\num{0.712(2)} & \num{1.50(5)} \\
\num{0.946(3)} & \num{2.00(6)} \\
\num{1.183(3)} & \num{2.50(7)} \\
\num{1.423(3)} & \num{3.00(7)} \\
\num{1.651(3)} & \num{3.50(8)} \\
\num{1.885(3)} & \num{4.00(8)} \\
\num{2.11(2)} & \num{4.50(9)} \\
\num{2.34(2)} & \num{5.00(9)} \\
\end{tabular}
\caption{Voltampérová chrakteristika vzorku}
\label{t:vodivost}
\end{tabulka}

\begin{graph}[htbp] 
\centering
\input{vod.tex}
\caption{Voltampérová charakteristika vzorku}
\label{g:vodivost}
\end{graph}


\begin{tabulka}[htbp]
\centering
\begin{tabular}{c||c|c|c|c||c|c|c|c}
& \multicolumn{4}{c||}{$I_{12}=\SI{2.50(6)}{\milli\ampere}$}  &  \multicolumn{4}{c}{$I_{12} = \SI{5.00(9)}{\milli\ampere}$} \\
$I_M (\si{\ampere})$ & $B (\si{\tesla})$ & $U_{56}^+ (\si{\milli\volt})$ & $U_{56}^- (\si{\milli\volt})$ & $U_H (\si{\milli\volt})$ & $B (\si{\tesla})$ & $U_{56}^+ (\si{\milli\volt})$ & $U_{56}^- (\si{\milli\volt})$  & $U_H (\si{\milli\volt})$ \\ \hline
\num{0.5} & \num{0.049} & \num{49} & \num{26} & \num{12} & \num{0.049} & \num{101} & \num{61}  & \num{20} \\
\num{1.0} & \num{0.098} & \num{59} & \num{14} & \num{23} & \num{0.098} & \num{123} & \num{38}  & \num{42} \\
\num{1.5} & \num{0.147} & \num{71} & \num{4} & \num{33} & \num{0.147} & \num{145} & \num{20}  & \num{63} \\
\num{2.0} & \num{0.196} & \num{82} & \num{-8} & \num{45} & \num{0.196} & \num{164} & \num{-2}  & \num{83} \\
\num{2.5} & \num{0.245} & \num{94} & \num{-17} & \num{55} & \num{0.245} & \num{187} & \num{-19}  & \num{103} \\
\num{3.0} & \num{0.294} & \num{104} & \num{-28} & \num{66} & \num{0.294} & \num{206} & \num{-41}  & \num{124} \\
\num{3.5} & \num{0.343} & \num{116} & \num{-37} & \num{76} & \num{0.343} & \num{230} & \num{-59}  & \num{145} \\
\num{4.0} & \num{0.392} & \num{126} & \num{-46} & \num{86} & \num{0.392} & \num{250} & \num{-77}  & \num{163} \\
\end{tabular}
\caption{Závislost Hallova napětí na magnetické indukci}
\label{t:hall}
\end{tabulka}

\begin{graph}[htbp]
\centering
\input{hall.tex}
\caption{Závislost Hallova napětí na magnetické indukci}
\label{g:hall}
\end{graph}