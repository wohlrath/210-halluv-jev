\section*{Výsledky měření}
Měření proběhlo při normálním tlaku a pokojové teplotě (přibližně \SI{22}{\degreeCelsius}).
Všechny uvedené nejistoty jsou standardní a v zápisu \num{10(1)} znamená číslo v závorce nejistotu v řádu poslední uvedené číslice.

Proud $I_{12}$ jsme měřili digitálním multimetrem MASTECH MY-68 a napětí $U_{34}$ a $U_{56}$ digitálním multimetrem METEX MXD 4660A.
Proud $I_M$ procházející elektromagnetem jsme měřili analogovým ampérmetrem s třídou přesnosti \num{0.5} a rozsahem \SI{6}{\ampere}.

Rozměry vzorku byly $l=\SI{6.000(5)}{\mm}$, $d=\SI{3.350(5)}{\mm}$ a $t=\SI{0.720(5)}{\mm}$.


Měrnou vodivost vzorku jsme určili $\sigma = \SI{5.28(5)}{\siemens\per\meter}$.
Naměřená voltampérová charakteristika je uvedena v tabulce \ref{t:vodivost} a zanesena do grafu \ref{g:vodivost}.



\begin{tabulka}[htbp]
\centering
\begin{tabular}{c|c}
$U_{34}$ (\si{\volt}) & $I_{12}$ (\si{\milli\ampere}) \\ \hline
\num{0.232(2)} & \num{0.50(4)} \\
\num{0.474(2)} & \num{1.00(5)} \\
\num{0.712(2)} & \num{1.50(5)} \\
\num{0.946(3)} & \num{2.00(6)} \\
\num{1.183(3)} & \num{2.50(7)} \\
\num{1.423(3)} & \num{3.00(7)} \\
\num{1.651(3)} & \num{3.50(8)} \\
\num{1.885(3)} & \num{4.00(8)} \\
\num{2.11(2)} & \num{4.50(9)} \\
\num{2.34(2)} & \num{5.00(9)} \\
\end{tabular}
\caption{Voltampérová chrakteristika vzorku}
\label{t:vodivost}
\end{tabulka}

\begin{graph}[htbp] 
\centering
\input{vod.tex}
\caption{Voltampérová charakteristika vzorku}
\label{g:vodivost}
\end{graph}

Magnetické pole buzené elektromagnetem mělo indukci
\begin{equation}
B(T)=\num{0.098} \cdot I_M(A)
\end{equation}

Změřili jsme Hallovu konstantu $R_H = \SI{0.061(1)}{\meter\cubed\per\ampere\per\second}$. Naměřené hodnoty jsou uvedeny v tabulce \ref{t:hall} a zaneseny do grafu \ref{g:hall}.
Hodnoty $U_{56}^+$, $U_{56}^-$ neuvádíme, pouze $U_H$, stejně tak místo $I_M$ uvádíme pouze $B$, tyto hodnoty jsou k nahlédnutí v záznamu z měření.


\begin{tabulka}[htbp]
\centering
\begin{tabular}{c|c||c|c}
\multicolumn{2}{c||}{$I_{12}=\SI{2.50(6)}{\milli\ampere}$}  &  \multicolumn{2}{c}{$I_{12} = \SI{5.00(9)}{\milli\ampere}$} \\
$B (\si{\tesla})$ & $U_H (\si{\milli\volt})$ & $B (\si{\tesla})$ & $U_H (\si{\milli\volt})$ \\ \hline
\num{0.049(3)} & \num{12(2)} & \num{0.049(3)} & \num{20(2)} \\
\num{0.098(3)} & \num{23(2)} & \num{0.098(3)} & \num{42(2)} \\
\num{0.147(3)} & \num{33(2)} & \num{0.147(3)} & \num{63(2)} \\
\num{0.196(3)} & \num{45(2)} & \num{0.196(3)} & \num{83(2)} \\
\num{0.245(3)} & \num{55(2)} & \num{0.245(3)} & \num{103(2)} \\
\num{0.294(3)} & \num{66(2)} & \num{0.294(3)} & \num{124(2)} \\
\num{0.343(3)} & \num{76(2)} & \num{0.343(3)} & \num{145(2)} \\
\num{0.392(3)} & \num{86(2)} & \num{0.392(3)} & \num{163(2)} \\
\end{tabular}
\caption{Měření Hallovy konstanty}
\label{t:hall}
\end{tabulka}

\begin{graph}[htbp]
\centering
\input{hall.tex}
\caption{Měření Hallovy konstanty}
\label{g:hall}
\end{graph}


Podle \eqref{eq:pohyblivost} jsme vypočítali Hallovskou pohyblivost $\mu = \SI{0.324(3)}{\ampere\second\squared\per\kg}$ a podle \eqref{eq:koncentrace} koncentraci nositelů náboje $n = \num{1.07(1)} \cdot \SI{e20}{\per\metre\cubed}$.